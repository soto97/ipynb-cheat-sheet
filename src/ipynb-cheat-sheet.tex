%% This file is encoded in utf-8
%%
\documentclass[12pt,landscape,english]{article}
\usepackage{babel}
\usepackage{multicol}
\usepackage{calc}
\usepackage{ifthen}
\usepackage[landscape]{geometry}
\usepackage{amsmath,amsthm,amsfonts,amssymb}
\usepackage{color,graphicx,overpic}
\usepackage{hyperref}

\usepackage[T1]{fontenc}
\usepackage[utf8]{inputenc}
\usepackage[defaultsans]{droidsans}
\renewcommand*\familydefault{\sfdefault}

\pdfinfo{
  /Title (iPython Cheat Sheet)
  /Creator (TeX)
  /Producer (pdfTeX 1.40.0)
  /Author (Seamus)
  /Subject (Example)
  /Keywords (pdflatex, latex,pdftex,tex)}

% This sets page margins to .5 inch if using letter paper, and to 1cm
% if using A4 paper. (This probably isn't strictly necessary.)
% If using another size paper, use default 1cm margins.
\ifthenelse{\lengthtest { \paperwidth = 11in}}
    { \geometry{top=.5in,left=.5in,right=.5in,bottom=.5in} }
    {\ifthenelse{ \lengthtest{ \paperwidth = 297mm}}
        {\geometry{top=1cm,left=1cm,right=1cm,bottom=1cm} }
        {\geometry{top=1cm,left=1cm,right=1cm,bottom=1cm} }
    }

% Turn off header and footer
\pagestyle{empty}

% Redefine section commands to use less space
\makeatletter
\renewcommand{\section}{\@startsection{section}{1}{0mm}%
                                {-1ex plus -.5ex minus -.2ex}%
                                {0.5ex plus .2ex}%x
                                {\normalfont\large\bfseries}}
\renewcommand{\subsection}{\@startsection{subsection}{2}{0mm}%
                                {-1explus -.5ex minus -.2ex}%
                                {0.5ex plus .2ex}%
                                {\normalfont\normalsize\bfseries}}
\renewcommand{\subsubsection}{\@startsection{subsubsection}{3}{0mm}%
                                {-1ex plus -.5ex minus -.2ex}%
                                {1ex plus .2ex}%
                                {\normalfont\small\bfseries}}
\makeatother

% Define BibTeX command
\def\BibTeX{{\rm B\kern-.05em{\sc i\kern-.025em b}\kern-.08em
    T\kern-.1667em\lower.7ex\hbox{E}\kern-.125emX}}

% Don't print section numbers
\setcounter{secnumdepth}{0}


\setlength{\parindent}{0pt}
\setlength{\parskip}{0pt plus 0.5ex}

%My Environments
%\newtheorem{example}[section]{Example}
% -----------------------------------------------------------------------

\begin{document}
\raggedright
\footnotesize
\begin{multicols}{3}

% multicol parameters
% These lengths are set only within the two main columns
%\setlength{\columnseprule}{0.25pt}
\setlength{\premulticols}{1pt}
\setlength{\postmulticols}{1pt}
\setlength{\multicolsep}{1pt}
\setlength{\columnsep}{2pt}
{
% title
\includegraphics[width=0.92\columnwidth]{imgs/ipynblogo}\\[0.47em]
\hfill\Large Cheat Sheet
\let\thefootnote\relax
\footnote{Originally Luis Martí --- \url{http://github.com/lmarti/ipybn-cheat-sheet}. Updated by Alejandro Soto --- \url{http://github.com/soto97/ipybn-cheat-sheet}.}
\rule{\columnwidth}{0.11pt}
}
\section{iPython notebook}
\begin{verbatim}
$ ipython notebook
\end{verbatim}

\section{General}
\begin{tabular}{@{}r@{\;}l@{}}
	\texttt{help}: &	access to docstrings and Python manuals \\
	\texttt{\%magic}: &	information on the 'magic' subsystem\\
	\texttt{\%alias}: & 	system command aliases \\
\end{tabular}

\section{Magic}

The magic function system provides a series of functions which allow you to
control the behavior of IPython itself, plus a lot of system-type
features.

\begin{tabular}{@{}r@{\;}l@{}}
	\texttt{\%matplotlib inline}: & plots inside the notebook \\
	\texttt{\%timeit}: & Time execution of code\\
\end{tabular}


%load:     Load code into the current frontend.
%lsmagic:    List currently available magic functions.

%matplotlib:     %matplotlib [gui]

%matplotlib:     %matplotlib [gui]

%quickref

%timeit:     Time execution of a Python statement or expression

%%html:     Render the cell as a block of HTML
%%javascript:     Run the cell block of Javascript code
%%latex:     Render the cell as a block of latex
\begin{tabular}{@{}r@{\;}l@{}}
	
	
\end{tabular}


\section{Keyboard shortcuts}
These commands are entered in the `Command Mode'. To enter Command Mode, either type \texttt{Esc} or \texttt{Ctrl-m}.

\subsection{General}
\begin{tabular}{@{}r@{\;}l@{}}
	\texttt{Tab}: &			completion in the local namespace \\
	\texttt{s}: &	save notebook \\
	\texttt{h}: &	show keyboard shortcuts \\
	\texttt{l}: & 	toggle line numbers \\
	\texttt{Ctrl-r}: & 		search history \\
\end{tabular}

\subsection{Editing}
\begin{tabular}{@{}r@{\;}l@{}}
	\texttt{x}: &	cut cell \\
    \texttt{c}: &	copy cell \\
	\texttt{v}: &	paste cell \\
	\texttt{d}: &	delete cell \\
	\texttt{z}: &	undo last cell deletion \\
\end{tabular}

\subsection{Cell management}
\begin{tabular}{@{}r@{\;}l@{}}
	\texttt{-}: &	split cell \\
	\texttt{a}: &	insert cell above $\uparrow$ \\
	\texttt{b}: &	insert cell below $\downarrow$ \\
	\texttt{k}: &	move cell up $\uparrow$ \\
	\texttt{j}: &	move cell down $\downarrow$ \\
	\texttt{p}: &	select previous $\uparrow$ \\
	\texttt{n}: &	select next $\downarrow$ \\
\end{tabular}

\subsection{Cell formatting}
\begin{tabular}{@{}r@{\;}l@{}}
	\texttt{y}: &	code cell \\
	\texttt{m}: &	markdown cell \\
	\texttt{t}: &	raw cell \\
	\texttt{1-6}: &	heading 1--6 cell \\
\end{tabular}

\subsection{Execution control}
\begin{tabular}{@{}r@{\;}l@{}}
	\texttt{Shift-Enter}: &	run cell \\
	\texttt{Ctrl-Enter}: &	run cell in-place \\
	\texttt{Alt-Enter}: &	run cell, insert below \\	
	\texttt{i}: &	interrupt kernel \\
	\texttt{.}: &	restart kernel \\
	\texttt{o}: &	toggle output \\
	\texttt{0}: &	toggle output scroll \\
\end{tabular}

\section{Cool packages}
\subsection{Execution control}
\begin{tabular}{@{}r@{\;}l@{}}
	Plotly: &	https://plot.ly/api/ipython \\
	\texttt{Ctrl-Enter}: &	run cell in-place \\
	\texttt{Alt-Enter}: &	run cell, insert below \\	
	\texttt{i}: &	interrupt kernel \\
	\texttt{.}: &	restart kernel \\
	\texttt{o}: &	toggle output \\
	\texttt{0}: &	toggle output scroll \\
\end{tabular}

% You can even have references
%\rule{0.3\linewidth}{0.25pt}
\scriptsize
%\bibliographystyle{abstract}q
%\bibliography{refFile}
\end{multicols}

\end{document}
